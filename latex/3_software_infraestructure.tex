\chapter{Software infraestructure}

For the use of Deep Learning have emerged numerous \textit{frameworks}. A deep learning framework allows us to build deep learning models more easily and quickly, without getting into all the details of the underlying algorithms. These are some of the most used today:
\begin{itemize}
\item \textbf{Tensorflow}: ofrece un API de bajo nivel que permite un control completo sobre los diseños de los modelos y también un API de alto nivel más simplificado pero con una funcionalidad limitada. Además permite la visualización del entrenamiento mediante la herramienta Tensorboard.
\item \textbf{Keras}: proporciona un API de alto nivel para uso de redes de neuronas. Puede correr sobre distintos \textit{backends} como Theano o Tensorflow y dispone modelos de redes pre-entrenadas que permiten crear una red de forma sencilla. Escrita en Python, ofrece un entorno amigable y modular.
\item \textbf{Caffe}: emplea una arquitectura C++/CUDA optimizada para uso en GPU y proporciona interfaces para Python o Matlab, por ejemplo. La definición del modelo se hace mediante Protobuf, formato creado por Google, creando una estructura de datos serializada. También dispone de modelos pre-entrenados e interfaz gráfico.
\item \textbf{PyTorch}:
\end{itemize}