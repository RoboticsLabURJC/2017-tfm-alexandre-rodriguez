\documentclass[a4paper, 12pt, oneside]{report}

\usepackage[utf8]{inputenc}
\usepackage{lmodern}
\usepackage{layout}
\usepackage{emptypage}
\usepackage{fancyhdr}
%\usepackage[Conny]{fncychap}
\usepackage{graphicx}
\usepackage{subfigure} % subfiguras
\usepackage{caption}
\usepackage{mathtools}
\usepackage{hyperref}
\usepackage[a4paper,top=3cm, bottom=3cm, inner=2.5cm, outer=2.5cm]{geometry}
\usepackage{listings}
\usepackage[british]{babel}
\usepackage{url}
\usepackage{float}
\usepackage{multirow}
\usepackage{rotating} 
\usepackage{color}
\usepackage{colortbl}
\usepackage{amsmath}
\usepackage{amssymb}
\usepackage[table]{xcolor}
%\usepackage[spanish]{babel}
\usepackage{algorithm}
\usepackage{algpseudocode}
\usepackage{comment}

\setlength{\parindent}{0pt}
\newcommand\abs[1]{\left|#1\right|}


%\usepackage[acronym, nonumberlist]{glossaries}
%\makeglossaries
%\include{0-Acronimos}

\makeatletter
\renewcommand{\@makeschapterhead}[1]{%
%  \vspace*{50\p@}%
  \vspace*{0\p@}%
  {\parindent \z@ \raggedright
    \normalfont
    \interlinepenalty\@M
    \Huge \bfseries  #1\par \nobreak
%    \vskip 40\p@
    \vskip 15\p@
  }}
\makeatother

\renewcommand{\baselinestretch}{1.4}
\setlength{\headheight}{16pt} 
\captionsetup{justification=justified}
\pretolerance=1000

\chead[]{}
\rhead[]{}
\renewcommand{\headrulewidth}{0.5pt}

\pagestyle{empty}

\title{Multiobject tracking using deep learning and tracking by detection}
\author{Alexandre Rodriguez Rendo}

\lstset{
	float=hbp,
	basicstyle=\ttfamily\small,
	columns=flexible,
	tabsize=4,
	frame=single,
	extendedchars=true,
	showspaces=false,
	showstringspaces=false,
	numbers=none,
	numberstyle=\tiny,
	breaklines=false,
	breakautoindent=true,
	captionpos=b
}
\setcounter{tocdepth}{4}
\setcounter{secnumdepth}{4}

\definecolor{lightgray}{gray}{0.9}

\begin{document}
%%%%%%%%%%%%%%% Portada %%%%%%%%%%%%%%%%%%%%
\begin{titlepage}
	
	\begin{center}
		\vspace*{7.7mm}
		\begin{center}
			\includegraphics[width=0.4\linewidth]{figures/logo.jpg}
		\end{center}
		\vspace{6.5mm}
		
		\fontsize{15.5}{14}\selectfont ESCUELA TÉCNICA SUPERIOR DE INGENIERÍA DE TELECOMUNICACIÓN
		\vspace{13mm}
		
		\fontsize{14}{14}\selectfont MASTER OFICIAL EN VISIÓN ARTIFICIAL 
		
		\vspace{70pt}
		
		\fontfamily{lmss}\fontsize{15.7}{14}\selectfont \textbf{Master thesis} 
		
		\vspace{25mm}
		\begin{huge}
			Multiobject tracking using deep learning and tracking by detection
		\end{huge}
		
		\vspace{25mm}
		
		\begin{large}
			Author: Alexandre Rodríguez Rendo
			
			Tutor: José María Cañas Plaza
						
			\vspace{10mm}
		\end{large}
		\begin{normalsize}
			Academic course 2018/2019		
		\end{normalsize}
		\vspace{10mm}
		
	\end{center}
	
\end{titlepage}

\pagebreak
\thispagestyle{empty}
\vspace*{12cm}

\begin{comment}
\begin{flushright}

\includegraphics[height=1.0cm]{figures/CC-BY-SA.png}

\vspace*{0.5cm}

\copyright 2017 Marcos Pieras Sagardoy

\vspace*{0.3cm}

Esta obra está distribuida bajo la licencia de 

``Reconocimiento-CompartirIgual 4.0 Internacional (CC BY-SA 4.0)''

de Creative Commons.

\vspace{0.2cm}

Para ver una copia de esta licencia, visite

http://creativecommons.org/licenses/by-sa/4.0/ o envíe

una carta a Creative Commons, 171 Second Street, Suite 300,

San Francisco, California 94105, USA.

\end{flushright}
\end{comment}

\pagenumbering{Roman}

%%%%%%%%%%%%%%% Agradecimientos %%%%%%%%%%%%
\chapter*{Acknowledgement}
\textit{Galician version}\\
En primeiro lugar quero dar as grazas ao meu titor Jose María polo apoio e axuda neste proxecto. Por outra banda, agradecer a amig@s, compañeir@s de estudo, de traballo, de vivenda... tod@s os que me acompañaron e animaron neste camiño.\\
Aos meus avós, os que están e os que xa non están pero seguen preto, por ser o maior exemplo de humanidade que un pode ser. Sen vós non sería nada do que son agora. Gracias de verdade.\\
A Cintia por ser á vez psicóloga, amiga e irmá.\\
Especialmente aos meus pais que me ensinaron a importancia de palabras como traballo, honestidade e valores. A partir do que vós me ensinastes fun capaz de poder aprender moitas cousas mais. Quérovos moito!\\
Seguramente me falte moita xente por mencionar pero creo que aquí está o máis importante.\\ \ \\
\textit{English version}\\
First of all I would like to thank my tutor Jose Maria for the support and help in this project. On the other hand, I would like to thank friends, colleagues, work colleagues, flat mates ... everyone who accompanied me and encouraged me on this path. \\
To my grandparents, those who are and those who are no longer but there are still close, for being the greatest example of humanity that one can be. Without you there I would not be what I am now. Thank you really. \\
To Cintia for being a psychologist, friend and sister at the same time. \\
Especially to my parents who taught me the importance of words such as work, honesty and values. From what you taught me I was able to learn many more things. I love you! \\
Surely I miss many people to mention but I think that here are the most important ones.


%%%%%%%%%%%%%%% Resumen %%%%%%%%%%%%%%%%%%%%
\include{0-Abstract}


\include{0-Resumen}

%%%%%%%%%%%%%%% Índices %%%%%%%%%%%%%%%%%%%%
\renewcommand{\tablename}{Tabla}
\renewcommand{\listtablename}{List of tables}
\tableofcontents

\cleardoublepage % Índice de figuras
\addcontentsline{toc}{chapter}{\listfigurename}
\listoffigures

\cleardoublepage % Índice de tablas
\addcontentsline{toc}{chapter}{List of tables}
\listoftables 


%%%%%%%%%%%%%%% Acronimos %%%%%%%%%%%%%%%%%%%%
%\renewcommand{\acronymname}{Acronyms}
%\cleardoublepage
%\addcontentsline{toc}{chapter}{Acronyms}
%\printglossary[type=\acronymtype]

\cleardoublepage
%%%%%%%%%%%%%%% Capítulos %%%%%%%%%%%%%%%%%%
\pagestyle{fancy}
\pagenumbering{arabic}
\setlength{\parindent}{6mm}

\lhead[]{CHAPTER \thechapter. Introduction}
\chapter{Introduction}\label{cap.introduccion}
\setlength{\parindent}{0pt}
The Multiple Object Tracking or \textit{MOT} is an important computer vision problem which continues to attract attention because of its potential in both the academic and commercial spheres. The real-world applications of the multiobject tracking are numerous including human-computer interaction, autonomous vehicles, robotics, video indexing, surveillance or security, among others. Many autonomous car projects are taking place globally which require solutions to various different problems including to keep an eye (not literally) to all other moving objects in the area where the car is located. The outputs from the tracking module are a basic input for other modules like maneuver planning and trajectory planning. Many multi-object tracking algorithms have been proposed to solve the problem of real-world traffic monitoring where algorithms have to deal with complex occlusion situations and difficult object matching.\\
In an era where human-computer interaction has become particularly important due to the quick development of touch display technology available in every smartphone, the hand is key and the object tracking is an important part of this area. For example, it is being used for recording the hand movements for its non-intrusive nature.
\\
The computer vision community have been making efforts in the past few decades but the MOT task is still open for improvement. One of the most studied tracking areas is the pedestrian tracking, mainly because the videos of pedestrians can be seen in a large number of applications with commercial potential. As some studies indicate [1], about the 70\% of the current research done in MOT is dedicated to pedestrians. The difficulty of MOT lies in various challenging situations that can occur such as variation of the illumination and the scale, target deformation or fast motion. Most of this challenges are common to Single Object Tracking (\textit{SOT}) but MOT also needs to solve two main tasks: determining the number of objects that usually vary over time and mantaining their identities.\\
In the \textit{Artificial Intelligence} era the multi-object tracking makes use also from the AI to improve the tracking algorithms. One of the most well known and currently growing subfields in AI is \textit{Deep Learning}. These techniques are being used for a broad range of applications such as object detection, object classification, biometrics or medical imaging, among others. In most cases, the Deep Learning has beaten the current State-of-the-Art is these areas.\\
Because of that, this master thesis will make use of Deep Learning to tackle the multi-object tracking problem.

\section{Objectives}
The objective of this master thesis is to build a multi-object tracker application using deep learning techniques. To achieve this, in this work we are going to use the \textit{Tracking by Detection} approach. This detections are going to be provided by deep learning neural networks.

\lhead[]{CHAPTER \thechapter. Objetivos}
\include{2-Objectives}

\lhead[]{CHAPTER \thechapter. Teorethic}
\include{3-Theoretic}

\lhead[]{CHAPTER \thechapter. Solution}
\include{4-Solution}

\lhead[]{CHAPTER \thechapter. Datasets and evaluation}
\include{5-Datasets}

\lhead[]{CHAPTER \thechapter. Experiments}
\include{6-Experiments}

\lhead[]{CHAPTER \thechapter. Conclusions}
\include{7-Conclusions}

%%%%%%%%%%%%%%% Bibliograía %%%%%%%%%%%%%%%
\lhead[]{BIBLIOGRAPHY}
\include{8-Bibliografia}

%\lhead[]{Annex}
%\include{9-Annexos}

\end{document}
